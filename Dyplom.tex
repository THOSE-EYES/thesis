%%%%%%%%%%%%%%%%%%%%%%%%%%%%%%%%%%%%%%%%%%%%%%%%%%%%%%%%%%%%%%%%%%%%%%%%%%%%%%%%
%  Zawartość: Główny plik szablonu pracy dyplomowej (magisterskiej/inżynierskiej). 
%  Opracował: Tomasz Kubik <tomasz.kubik@pwr.edu.pl>
%  Data: styczeń 2023
%  Wersja: 0.9
%  Wymagania: kompilator pdflatex
%%%%%%%%%%%%%%%%%%%%%%%%%%%%%%%%%%%%%%%%%%%%%%%%%%%%%%%%%%%%%%%%%%%%%%%%%%%%%%%%

\documentclass[a4paper,onecolumn,oneside,12pt,extrafontsizes]{memoir}
%  W celu przygotowania wydruku do archiwum można:
%  a) przygotować pdf, w którym dwie strony zostaną wstawione na jedną fizyczną stronę i taki dokument wydrukować dwustronnie (podejście zalecane)
%
%   Taki dokument można przygotować poprzez
%   - wydruk z Adobe Acrobat Reader z opcją "Wiele" - sekcja "Rozmiar i obsługa stron"
%   - wykorzystanie narzędzi psutils
%
%      Windows (zakładając, że w dystrybucji MiKTeX jest pakiet miktex-psutils-bin-x64-2.9):
%        "c:\Program Files\MiKTeX 2.9\miktex\bin\x64\pdf2ps.exe" Dyplom.pdf Dyplom.ps
%        "c:\Program Files\MiKTeX 2.9\miktex\bin\x64\psnup.exe" -2 Dyplom.ps Dyplom2.ps
%        "c:\Program Files\MiKTeX 2.9\miktex\bin\x64\ps2pdf.exe" Dyplom2.ps Dyplom2.pdf
%        Del Dyplom2.ps Dyplom.ps
%
%     Linux:
%        pdf2ps Dyplom.pdf - | psnup -2 | ps2pdf - Dyplom2.pdf
%
%  b) przekomplilować dokument zmniejszając czcionkę (podejście niezalecane, bo zmienia formatowanie dokumentu)
%
%    Do tego wystarczy posłużyć się poniższymi komendami (zamiast documentclass z pierwszej linijki):
%   \documentclass[a4paper,onecolumn,twoside,10pt]{memoir} 
%   \renewcommand{\normalsize}{\fontsize{8pt}{10pt}\selectfont}

%\usepackage[cp1250]{inputenc} % Proszę zostawić, jeśli kodowanie edytowanych plików to cp1250 
\usepackage[utf8]{inputenc} % Proszę użyć zamiast powyższego, jeśli kodowanie edytowanych plików to UTF8
\usepackage[T1]{fontenc}
\usepackage[english,polish]{babel} % Tutaj ważna jest kolejność atrybutów (dla pracy po polsku polish powinno być na końcu)
%\DisemulatePackage{setspace}
\usepackage{setspace}
\usepackage{color,calc}
%\usepackage{soul} % pakiet z komendami do podkreślania, przekreślania, podświetlania tekstu (raczej niepotrzebny)
\usepackage{ebgaramond} % pakiet z czcionkami garamond, potrzebny tylko do strony tytułowej, musi wystąpić przed pakietem tgtermes

%% Aby uzyskać polskie literki w pdfie (a nie zlepki) korzystamy z pakietu czcionek tgterms. 
%% W pakiecie tym są zdefiniowane klony czcionek Times o kształtach: normalny, pogrubiony, italic, italic pogrubiony.
%% W pakiecie tym brakuje czcionki o kształcie: slanted (podobny do italic). 
%% Jeśli w dokumencie gdzieś zostanie zastosowana czcionka slanted (np. po użyciu komendy \textsl{}), to
%% latex dokona podstawienia na czcionkę standardową i zgłosi to w ostrzeżeniu (warningu).
%% Ponadto tgtermes to czcionka do tekstu. Wszelkie matematyczne wzory będą sformatowane domyślną czcionką do wzorów.
%% Jeśli wzory mają być sformatowane z wykorzystaniem innych czcionek, trzeba to jawnie zadeklarować.

%% Po zainstalowaniu pakietu tgtermes może będzie trzeba zauktualizować informacje 
%% o dostępnych fontach oraz mapy. Można to zrobić z konsoli (jako administrator)
%% initexmf --admin --update-fndb
%% initexmf --admin --mkmaps

\usepackage{tgtermes}   
\renewcommand*\ttdefault{txtt}


%%%%%%%%%%%%%%%%%%%%%%%%%%%%%%%%%%%%%%%%%%%%%%%%%%%%%%%%%%%%%%%%%%%%%%%%%%%%%%%%
%% Ustawienia odpowiedzialne za sposób łamania dokumentu
%% i ułożenie elementów pływających
%%%%%%%%%%%%%%%%%%%%%%%%%%%%%%%%%%%%%%%%%%%%%%%%%%%%%%%%%%%%%%%%%%%%%%%%%%%%%%%%
%\hyphenpenalty=10000		% nie dziel wyrazów zbyt często
\clubpenalty=10000      % kara za sierotki
\widowpenalty=10000     % nie pozostawiaj wdów
%\brokenpenalty=10000		% nie dziel wyrazów między stronami - trzeba było wyłączyć, bo nie łamały się linie w lstlisting
%\exhyphenpenalty=999999		% nie dziel słów z myślnikiem - trzeba było wyłączyć, bo nie łamały się linie w lstlisting
\righthyphenmin=3			  % dziel minimum 3 litery

%\tolerance=4500
%\pretolerance=250
%\hfuzz=1.5pt
%\hbadness=1450

\renewcommand{\topfraction}{0.95}
\renewcommand{\bottomfraction}{0.95}
\renewcommand{\textfraction}{0.05}
\renewcommand{\floatpagefraction}{0.35}

%%%%%%%%%%%%%%%%%%%%%%%%%%%%%%%%%%%%%%%%%%%%%%%%%%%%%%%%%%%%%%%%%%%%%%%%%%%%%%%%
%%  Ustawienia rozmiarów: tekstu, nagłówka i stopki, marginesów
%%  dla dokumentów klasy memoir 
%%%%%%%%%%%%%%%%%%%%%%%%%%%%%%%%%%%%%%%%%%%%%%%%%%%%%%%%%%%%%%%%%%%%%%%%%%%%%%%%
\setlength{\headsep}{10pt} 
\setlength{\headheight}{13.6pt} % wartość baselineskip dla czcionki 11pt tj. \small wynosi 13.6pt
\setlength{\footskip}{\headsep+\headheight}
\setlength{\uppermargin}{\headheight+\headsep+1cm}
\setlength{\textheight}{\paperheight-\uppermargin-\footskip-1.5cm}
\setlength{\textwidth}{\paperwidth-5cm}
\setlength{\spinemargin}{2.5cm}
\setlength{\foremargin}{2.5cm}
\setlength{\marginparsep}{2mm}
\setlength{\marginparwidth}{2.3mm}
%\settrimmedsize{297mm}{210mm}{*}
%\settrims{0mm}{0mm}	
\checkandfixthelayout[fixed] % konieczne, aby się dobrze wszystko poustawiało
%%%%%%%%%%%%%%%%%%%%%%%%%%%%%%%%%%%%%%%%%%%%%%%%%%%%%%%%%%%%%%%%%%%%%%%%%%%%%%%%
%%  Ustawienia odległości linii, wcięć, odstępów
%%%%%%%%%%%%%%%%%%%%%%%%%%%%%%%%%%%%%%%%%%%%%%%%%%%%%%%%%%%%%%%%%%%%%%%%%%%%%%%%
\linespread{1}
%\linespread{1.241}
\setlength{\parindent}{14.5pt}


\usepackage{multicol} % pakiet umożliwiający stworzenie wielokolumnowego tekstu
%%%%%%%%%%%%%%%%%%%%%%%%%%%%%%%%%%%%%%%%%%%%%%%%%%%%%%%%%%%%%%%%%%%%%%%%%%%%%%%%
%% Pakiety do formatowania tabel
%%%%%%%%%%%%%%%%%%%%%%%%%%%%%%%%%%%%%%%%%%%%%%%%%%%%%%%%%%%%%%%%%%%%%%%%%%%%%%%%
\usepackage{tabularx}
% Proszę używać tylko tabularx. Innych pakietów proszę nie stosować !!!
% Dokument na pewno da się zredagować bez ich użycia.
%\usepackage{longtable}
%\usepackage{ltxtable}
%\usepackage{tabulary}

%%%%%%%%%%%%%%%%%%%%%%%%%%%%%%%%%%%%%%%%%%%%%%%%%%%%%%%%%%%%%%%%%%%%%%%%%%%%%%%%
%% Pakiet do wstawiania fragmentów kodu
%%%%%%%%%%%%%%%%%%%%%%%%%%%%%%%%%%%%%%%%%%%%%%%%%%%%%%%%%%%%%%%%%%%%%%%%%%%%%%%%
\usepackage{listings} 
\usepackage{xpatch}
\makeatletter
\xpatchcmd\l@lstlisting{1.5em}{0em}{}{}
\makeatother
% Pakiet dostarcza otoczenia lstlisting. Jest ono wysoce konfigurowalne. 
% Konfigurować można indywidualnie każdy z listingów lub globalnie, w poleceniu \lstset{}.

% Zalecane jest, by kod źródłowy był wyprowadzany z użyciem czcionki maszynowej \ttfamily
% Ponieważ kod źródłowy, nawet po obcięciu do interesujących fragmentów, bywa obszerny, należy zmniejszyć czcionkę.
% Zalecane jest \small (dla krótkich fragmentów) oraz \footnotesize (dla dłuższych fragmentów).

% Ponadto podczas konfiguracji można zadeklarować sposób numerowania linii. Numerowanie linii zalecane jest jednak 
% tylko w przypadkach, gdy w redagowanym tekście znajdują się jakieś odwołania do konkretnych linii.
% Jeśli takich odwołań nie ma, numerowanie linii jest zbędne. Proszę wtedy go nie stosować.
% Przy włączaniu numerowania linii należy zwrócić uwagę na to, gdzie pojawią się te numery.
% Bez zmiany dodatkowych parametrów pojawiają się one na marginesie strony (co jest niepożądane).

\lstset{
  basicstyle=\small\ttfamily, % lub basicstyle=\footnotesize\ttfamily
  %%columns=fullflexible,
	%%showstringspaces=false,
	%%showspaces=false,
  breaklines=true,
  postbreak=\mbox{\textcolor{red}{$\hookrightarrow$}\space}, 
  %%numbers=left,  % ta i poniższe linie dotyczą ustawienia numerowania i sposobu jego wyprowadzania
  %%firstnumber=1, 
  %%numberfirstline=true, 
	%%xleftmargin=17pt,
  %%framexleftmargin=17pt,
  %%framexrightmargin=5pt,
  %%framexbottommargin=4pt,
	belowskip=.5\baselineskip,
	literate={\_}{{\_\allowbreak}}1 % ta deklaracja przydaje się, jeśli na listingu mają być łamane nazwy zawierające podkreślniki
}

% Jeśli edytowany plik nie jest w kodowaniu cp1250, to jest problem z polskimi znakami występującymi we wstawianym kodzie.
% Dlatego podczas pracy na plikach w kodowaniu UTF8 trzeba zadeklarować mapowanie jak niżej (wystarczy odmarkować).
% Niestety, jak się zastosuje to mapowanie mogą pojawić się problemy z podświetlaniem składni (patrz dalej).
\lstset{literate=%-
{ą}{{\k{a}}}1 {ć}{{\'c}}1 {ę}{{\k{e}}}1 {ł}{{\l{}}}1 {ń}{{\'n}}1 {ó}{{\'o}}1 {ś}{{\'s}}1 {ż}{{\.z}}1 {ź}{{\'z}}1 {Ą}{{\k{A}}}1 {Ć}{{\'C}}1 {Ę}{{\k{E}}}1 {Ł}{{\L{}}}1 {Ń}{{\'N}}1 {Ó}{{\'O}}1 {Ś}{{\'S}}1 {Ż}{{\.Z}}1 {Ź}{{\'Z}}1 
    {Ö}{{\"O}}1
    {Ä}{{\"A}}1
    {Ü}{{\"U}}1
    {ß}{{\ss}}1
    {ü}{{\"u}}1
    {ä}{{\"a}}1
    {ö}{{\"o}}1
    {~}{{\textasciitilde}}1
		{—}{{{\textemdash} }}1
}%{\ \ }{{\ }}1}


%% lstlisting pozwala na ostylowania podświetlania składni wybranych języków.
%% Działa to na zasadzie zdefiniowania słów kluczowych oraz sposobu ich wyświetlania.
%% Ponieważ jest to prosty mechanizm, czasem trudno osiągnąć takie efekty, jakie dają narzędzia IDE. 
%% Jednak w większości przypadku osiągane rezutlaty są zadowalające.


%% lstlisting obsługuje domyślnie kilka najpopularniejszych języków.
%%\lstloadlanguages{% Check Dokumentation for further languages ...
%%C,
%%C++,
%%csh,
%%Java
%%}
%% Inne języki muszą być dodefiniowane. Poniżej podano przykłady definicji języków i styli.

\definecolor{lightgray}{rgb}{.9,.9,.9}
\definecolor{darkgray}{rgb}{.4,.4,.4}
\definecolor{purple}{rgb}{0.65, 0.12, 0.82}
\definecolor{javared}{rgb}{0.6,0,0} % for strings
\definecolor{javagreen}{rgb}{0.25,0.5,0.35} % comments
\definecolor{javapurple}{rgb}{0.5,0,0.35} % keywords
\definecolor{javadocblue}{rgb}{0.25,0.35,0.75} % javadoc
 
\lstdefinelanguage{JavaScript}{ 
	keywords={typeof, new, true, false, catch, function, return, null, catch, switch, var, if, in, while, do, else, case, break},
	keywordstyle=\color{blue}\bfseries,
	ndkeywords={class, export, boolean, throw, implements, import, this},
	ndkeywordstyle=\color{darkgray}\bfseries,
	identifierstyle=\color{black},
	sensitive=false,
	comment=[l]{//},
	morecomment=[s]{/*}{*/},
	commentstyle=\color{purple}\ttfamily,
	stringstyle=\color{red}\ttfamily,
	morestring=[b]',
	morestring=[b]"
}
\lstdefinestyle{JavaScriptStyle}{
	language=JavaScript,
	commentstyle=\color{javagreen}, % niestety, jeśli w linii komentarza pojawią się słowa kluczowe, to zostaną pokolorowane
	backgroundcolor=,%\color{lightgray}, % można ustwić kolor tła, ale jest to niezalecane
	extendedchars=true,
	basicstyle=\footnotesize\ttfamily,
	showstringspaces=false,
	showspaces=false,
	numbers=none,%left,
	numberstyle=\footnotesize,
	numbersep=9pt,
	tabsize=2,
	breaklines=true,
	showtabs=false,
	captionpos=t
}

\lstdefinestyle{JavaStyle}{
basicstyle=\footnotesize\ttfamily,
keywordstyle=\color{javapurple}\bfseries,
stringstyle=\color{javared},
commentstyle=\color{javagreen},
morecomment=[s][\color{javadocblue}]{/**}{*/},
numbers=none,%left,
numberstyle=\tiny\color{black},
stepnumber=2,
numbersep=10pt,
tabsize=4,
showspaces=false,
showstringspaces=false,
captionpos=t
}

\definecolor{pblue}{rgb}{0.13,0.13,1}
\definecolor{pgreen}{rgb}{0,0.5,0}
\definecolor{pred}{rgb}{0.9,0,0}
\definecolor{pgrey}{rgb}{0.46,0.45,0.48}
\definecolor{dark-grey}{rgb}{0.4,0.4,0.4}
% styl json
\newcommand\JSONnumbervaluestyle{\color{blue}}
\newcommand\JSONstringvaluestyle{\color{red}}

\newif\ifcolonfoundonthisline

\makeatletter

\lstdefinestyle{json-style}  
{
	showstringspaces    = false,
	keywords            = {false,true},
	alsoletter          = 0123456789.,
	morestring          = [s]{"}{"},
	stringstyle         = \ifcolonfoundonthisline\JSONstringvaluestyle\fi,
	MoreSelectCharTable =%
	\lst@DefSaveDef{`:}\colon@json{\processColon@json},
	basicstyle          = \footnotesize\ttfamily,
	keywordstyle        = \ttfamily\bfseries,
	numbers				= left, % zakomentować, jeśli numeracja linii jest niepotrzebna
	numberstyle={\footnotesize\ttfamily\color{dark-grey}},
	xleftmargin			= 2em % zakomentować, jeśli numeracja linii jest niepotrzebna
}

\newcommand\processColon@json{%
	\colon@json%
	\ifnum\lst@mode=\lst@Pmode%
	\global\colonfoundonthislinetrue%
	\fi
}

\lst@AddToHook{Output}{%
	\ifcolonfoundonthisline%
	\ifnum\lst@mode=\lst@Pmode%
	\def\lst@thestyle{\JSONnumbervaluestyle}%
	\fi
	\fi
	\lsthk@DetectKeywords% 
}

\lst@AddToHook{EOL}%
{\global\colonfoundonthislinefalse}

\makeatother

%%\definecolor{red}{rgb}{0.6,0,0} % for strings
%%\definecolor{blue}{rgb}{0,0,0.6}
%%\definecolor{green}{rgb}{0,0.8,0}
%%\definecolor{cyan}{rgb}{0.0,0.6,0.6}
%%
%%\lstdefinestyle{sqlstyle}{
%%language=SQL,
%%basicstyle=\footnotesize\ttfamily, 
%%numbers=left, 
%%numberstyle=\tiny, 
%%numbersep=5pt, 
%%tabsize=2, 
%%extendedchars=true, 
%%breaklines=true, 
%%showspaces=false, 
%%showtabs=true, 
%%xleftmargin=17pt,
%%framexleftmargin=17pt,
%%framexrightmargin=5pt,
%%framexbottommargin=4pt,
%%keywordstyle=\color{blue}, 
%%commentstyle=\color{green}, 
%%stringstyle=\color{red}, 
%%}
%%
%%\lstdefinestyle{sharpcstyle}{
%%language=[Sharp]C,
%%basicstyle=\footnotesize\ttfamily, 
%%numbers=left, 
%%numberstyle=\tiny, 
%%numbersep=5pt, 
%%tabsize=2, 
%%extendedchars=true, 
%%breaklines=true, 
%%showspaces=false, 
%%showtabs=true, 
%%xleftmargin=17pt,
%%framexleftmargin=17pt,
%%framexrightmargin=5pt,
%%framexbottommargin=4pt,
%%morecomment=[l]{//}, %use comment-line-style!
%%morecomment=[s]{/*}{*/}, %for multiline comments
%%showstringspaces=false, 
%%morekeywords={  abstract, event, new, struct,
                %%as, explicit, null, switch,
                %%base, extern, object, this,
                %%bool, false, operator, throw,
                %%break, finally, out, true,
                %%byte, fixed, override, try,
                %%case, float, params, typeof,
                %%catch, for, private, uint,
                %%char, foreach, protected, ulong,
                %%checked, goto, public, unchecked,
                %%class, if, readonly, unsafe,
                %%const, implicit, ref, ushort,
                %%continue, in, return, using,
                %%decimal, int, sbyte, virtual,
                %%default, interface, sealed, volatile,
                %%delegate, internal, short, void,
                %%do, is, sizeof, while,
                %%double, lock, stackalloc,
                %%else, long, static,
                %%enum, namespace, string},
%%keywordstyle=\color{cyan},
%%identifierstyle=\color{red},
%%stringstyle=\color{blue}, 
%%commentstyle=\color{green},
%%}



%%%%%%%%%%%%%%%%%%%%%%%%%%%%%%%%%%%%%%%%%%%%%%%%%%%%%%%%%%%%%%%%%%%%%%%%%%%%%%%%
%%  Pakiety i komendy zastosowane tylko do zamieszczenia informacji o użytych komendach i fontach w tym szablonie.
%%  Normalnie nie są one potrzebne. Proszę poniższe deklaracje zamarkować podczas redakcji pracy !!!!
%%%%%%%%%%%%%%%%%%%%%%%%%%%%%%%%%%%%%%%%%%%%%%%%%%%%%%%%%%%%%%%%%%%%%%%%%%%%%%%%
\usepackage{memlays}     % extra layout diagrams, zastosowane w szblonie do 'debuggowania', używa pakietu layouts
%\usepackage{layouts}
\usepackage{printlen} % pakiet do wyświetlania wartości zdefiniowanych długości, stosowany do 'debuggowania'
\usepackage{enumitem} % pakiet do numerowania 1.1 1.2 w sekcji enumrate
\uselengthunit{pt}
\makeatletter
\newcommand{\showFontSize}{\f@size pt} % makro wypisujące wielkość bieżącej czcionki
\makeatother
% do pokazania ramek można byłoby użyć:
%\usepackage{showframe} 

%%%%%%%%%%%%%%%%%%%%%%%%%%%%%%%%%%%%%%%%%%%%%%%%%%%%%%%%%%%%%%%%%%%%%%%%%%%%%%%%
%%  Formatowanie list wyliczeniowych, wypunktowań i własnych otoczeń
%%%%%%%%%%%%%%%%%%%%%%%%%%%%%%%%%%%%%%%%%%%%%%%%%%%%%%%%%%%%%%%%%%%%%%%%%%%%%%%%

% Domyślnie wypunktowania mają zadeklarowane znaki, które nie występują w tgtermes
% Aby latex nie podstawiał w ich miejsca znaków z czcionki standardowej można zrobić podstawienie:
%    \DeclareTextCommandDefault{\textbullet}{\ensuremath{\bullet}}
%    \DeclareTextCommandDefault{\textasteriskcentered}{\ensuremath{\ast}}
%    \DeclareTextCommandDefault{\textperiodcentered}{\ensuremath{\cdot}}
% Jednak jeszcze lepszym pomysłem jest zdefiniowanie otoczeń z wykorzystaniem enumitem
\usepackage{enumitem} % pakiet pozwalający zarządzać formatowaniem list wyliczeniowych
\setlist{noitemsep,topsep=4pt,parsep=0pt,partopsep=4pt,leftmargin=*} % zadeklarowane parametry pozwalają uzyskać 'zwartą' postać wypunktowania bądź wyliczenia
\setenumerate{labelindent=0pt,itemindent=0pt,leftmargin=!,label=\arabic*.} % można zmienić \arabic na \alph, jeśli wyliczenia mają być z literkami
\setlistdepth{4} % definiujemy głębokość zagnieżdżenia list wyliczeniowych do 4 poziomów
\setlist[itemize,1]{label=$\bullet$}  % definiujemy, jaki symbol ma być użyty w wyliczeniu na danym poziomie
\setlist[itemize,2]{label=\normalfont\bfseries\textendash}
\setlist[itemize,3]{label=$\ast$}
\setlist[itemize,4]{label=$\cdot$}
\renewlist{itemize}{itemize}{4}

%%%http://tex.stackexchange.com/questions/29322/how-to-make-enumerate-items-align-at-left-margin
%\renewenvironment{enumerate}
%{
%\begin{list}{\arabic{enumi}.}
%{
%\usecounter{enumi}
%%\setlength{\itemindent}{0pt}
%%\setlength{\leftmargin}{1.8em}%{2zw} % 
%%\setlength{\rightmargin}{0zw} %
%%\setlength{\labelsep}{1zw} %
%%\setlength{\labelwidth}{3zw} % 
%\setlength{\topsep}{6pt}%
%\setlength{\partopsep}{0pt}%
%\setlength{\parskip}{0pt}%
%\setlength{\parsep}{0em} % 
%\setlength{\itemsep}{0em} % 
%%\setlength{\listparindent}{1zw} % 
%}
%}{
%\end{list}
%}

\makeatletter
\renewenvironment{quote}{
	\begin{list}{}
	{
	\setlength{\leftmargin}{1em}
	\setlength{\topsep}{0pt}%
	\setlength{\partopsep}{0pt}%
	\setlength{\parskip}{0pt}%
	\setlength{\parsep}{0pt}%
	\setlength{\itemsep}{0pt}
	}
	}{
	\end{list}}
\makeatother

%%%%%%%%%%%%%%%%%%%%%%%%%%%%%%%%%%%%%%%%%%%%%%%%%%%%%%%%%%%%%%%%%%%%%%%%%%%%%%%%
%%  Pakiet i komendy do generowania indeksu 
%% (ważne, by pojawiły się przed pakietem hyperref)
%%%%%%%%%%%%%%%%%%%%%%%%%%%%%%%%%%%%%%%%%%%%%%%%%%%%%%%%%%%%%%%%%%%%%%%%%%%%%%%%
% pdftex jest w stanie wygenerować indeks (czyli spis haseł z referencjami do stron, na których te hasła się pojawiły).
% Generalnie z indeksem jest sporo problemów, zwłaszcza, gdy pojawiają się polskie literki.
% Trzeba wtedy korzystać z xindy.
% Zwykle w pracach dyplomowych indeksy nie są wykorzystywane. Dlatego są zamarkowane.
%\DisemulatePackage{imakeidx}
%\usepackage[makeindex,noautomatic]{imakeidx} % tutaj mówimy, żeby indeks nie generował się automatycznie, 
%\makeindex
%
%\makeatletter
%%%%\renewenvironment{theindex}
							 %%%%{\vskip 10pt\@makeschapterhead{\indexname}\vskip -3pt%
								%%%%\@mkboth{\MakeUppercase\indexname}%
												%%%%{\MakeUppercase\indexname}%
								%%%%\vspace{-3.2mm}\parindent\z@%
								%%%%\renewcommand\subitem{\par\hangindent 16\p@ \hspace*{0\p@}}%%
								%%%%\phantomsection%
								%%%%\begin{multicols}{2}
								%%%%%\thispagestyle{plain}
								%%%%\parindent\z@                
								%%%%%\parskip\z@ \@plus .3\p@\relax
								%%%%\let\item\@idxitem}
							 %%%%{\end{multicols}\clearpage}
%%%%
%\makeatother




%%%%%%%%%%%%%%%%%%%%%%%%%%%%%%%%%%%%%%%%%%%%%%%%%%%%%%%%%%%%%%%%%%%%%%%%%%%%%%%%
%%  Sprawy metadanych w wynikowym pdf, hyperlinków itp.
%%%%%%%%%%%%%%%%%%%%%%%%%%%%%%%%%%%%%%%%%%%%%%%%%%%%%%%%%%%%%%%%%%%%%%%%%%%%%%%%
% Szablon przygotowano głównie dla pdflatex. Specyficzne komendy dla pdf-owej kompilacj wstawiono 
% w instrukcję warunkową dostarczaną przez pakiet ifpdf 
% Jeśli metadane zawierają przecinki lub średniki, domyślnie metadane te otaczane są apostrofami.
% Piszą o tym na stronie: https://tex.stackexchange.com/questions/3708/hyperref-enquotes-metadata
% Aby pozbyć się tych apostrofów użyto pakietu hyperxmp (ładującego kilka innych pakietów)
\usepackage{ifpdf}
%\newif\ifpdf \ifx\pdfoutput\undefined
%\pdffalse % we are not running PDFLaTeX
%\else
%\pdfoutput=1 % we are running PDFLaTeX
%\pdftrue \fi
\ifpdf
 \usepackage{datetime2} % INFO: pakiet potrzeby do uzyskania i sformatowania daty 
 \usepackage[pdftex,bookmarks,breaklinks,unicode]{hyperref}
 \usepackage{hyperxmp}
 \usepackage[pdftex]{graphicx}
 \DeclareGraphicsExtensions{.pdf,.jpg,.mps,.png} % po zadeklarowaniu rozszerzeń można będzie wstawiać pliki z grafiką bez konieczności podawania tych rozszerzeń w ich nazwach
\pdfcompresslevel=9
\pdfoutput=1

% Dobrze przygotowany dokument pdf to taki, który zawiera metadane.
% Poniżej zadeklarowano pola metadanych, jakie będą włączone do dokumentu pdf.
% Można je zmodyfikować w zależności od potrzeb
\makeatletter
\AtBeginDocument{  
  \hypersetup{
	pdfinfo={
    Title = {\@title},
    Author = {\@author},
    Subject={Praca dyplomowa \ifMaster magisterska\else inżynierska\fi},  
    Keywords={\@kvpl}, 
		Producer={}, 
	  CreationDate= {}, % należy wstawiać zgodnie ze składnią: {D:yyyymmddhhmmss}, np. D:20210208175600
    ModDate={\pdfcreationdate},   % data modyfikacji będzie datą kompilacji
		Creator={pdftex},
	}}
}
\pdftrailerid{} %Remove ID
\pdfsuppressptexinfo15 %Suppress PTEX.Fullbanner and info of imported PDFs
\makeatother
\else             % jeśli kompilacja jest inna niż pdflatex
\usepackage{graphicx}
\DeclareGraphicsExtensions{.eps,.ps,.jpg,.mps,.png}
\fi
\sloppy

% INFO: dodane by lepiej łamać urle 
\def\UrlBreaks{\do\/\do-\do_} 
% INFO: choć można zadeklarować foldery, w jakich pojawiać się mają pliki z grafiką, zaleca się jednak, by tego nie robić
%\graphicspath{{rys01/}{rys02/}}  


%%%%%%%%%%%%%%%%%%%%%%%%%%%%%%%%%%%%%%%%%%%%%%%%%%%%%%%%%%%%%%%%%%%%%%%%%%%%%%%%
%%  Formatowanie dokumentu
%%%%%%%%%%%%%%%%%%%%%%%%%%%%%%%%%%%%%%%%%%%%%%%%%%%%%%%%%%%%%%%%%%%%%%%%%%%%%%%%
% INFO: Deklaracja głębokościu numeracji
\setcounter{secnumdepth}{2}
\setcounter{tocdepth}{2}
\setsecnumdepth{subsection} 
% INFO: Dodanie kropek po numerach sekcji
\makeatletter
\def\@seccntformat#1{\csname the#1\endcsname.\quad}
\def\numberline#1{\hb@xt@\@tempdima{#1\if&#1&\else.\fi\hfil}}
\makeatother
% INFO: Numeracja rozdziałów i separatory
\renewcommand{\chapternumberline}[1]{#1.\quad}
\renewcommand{\cftchapterdotsep}{\cftdotsep}


%\usepackage{etoolbox} % odstępy w spisie treści (jeden ze sposobów ustawiania)
%%\makeatletter
%%\pretocmd{\chapter}{\addtocontents{toc}{\protect\addvspace{-1\p@}}}{}{}
%%\pretocmd{\section}{\addtocontents{toc}{\protect\addvspace{-1\p@}}}{}{}
%%\pretocmd{\subsection}{\addtocontents{toc}{\protect\addvspace{-1\p@}}}{}{}
%%\makeatother

\makeatletter % odstępy w spisie pomiędzy rozdziałami
\renewcommand*{\insertchapterspace}{%
  \addtocontents{lof}{\protect\addvspace{3pt}}%
  \addtocontents{lot}{\protect\addvspace{3pt}}%
	\addtocontents{toc}{\protect\addvspace{3pt}} %
  \addtocontents{lol}{\protect\addvspace{3pt}}}
\makeatother 


\setlength{\cftbeforechapterskip}{0pt} % odstępy w spisie treści przed rozdziałem, działa w korelacji z:
\renewcommand{\aftertoctitle}{\afterchaptertitle\vspace{-4pt}} % 
% https://stackoverflow.com/questions/3029271/latex-make-listoffigures-look-like-listoftables-or-lstlistoflistings
%\renewcommand{\memchapinfo}[4]{%
%  \addtocontents{lol}{\protect\addvspace{10pt}}
%}

%\cftsetindents{section}{1.5em}{2.3em}

%\setbeforesecskip{10pt plus 0.5ex}%{-3.5ex \@plus -1ex \@minus -.2ex}
%\setaftersecskip{10pt plus 0.5ex}%\onelineskip}
%\setbeforesubsecskip{8pt plus 0.5ex}%{-3.5ex \@plus -1ex \@minus -.2ex}
%\setaftersubsecskip{8pt plus 0.5ex}%\onelineskip}
%\setlength\floatsep{6pt plus 2pt minus 2pt} 
%\setlength\intextsep{12pt plus 2pt minus 2pt} 
%\setlength\textfloatsep{12pt plus 2pt minus 2pt} 

% Ustawienie odstępu od góry w nienumerowanych rozdziałach oraz wykazach:
% Spis treści, Spis tabel, Spis rysunków, Indeks rzeczowy
%\newlength{\linespace}
%\setlength{\linespace}{-\beforechapskip-\topskip+\headheight+\topsep}
%%%\makechapterstyle{noNumbered}{%
%%%\renewcommand\chapterheadstart{\vspace*{\linespace}}
%%%}
%% powyższa komenda załatwia to, co robią komendy poniższe dla spisów
%\renewcommand*{\tocheadstart}{\vspace*{\linespace}}
%\renewcommand*{\lotheadstart}{\vspace*{\linespace}}
%\renewcommand*{\lofheadstart}{\vspace*{\linespace}}


% INFO: Czcionka do podpisów tabel, rysunków, listingów
\captionnamefont{\small}
\captiontitlefont{\small}


% INFO: Sformatowanie podpisu nad dwukolumnowym listingiem
\newcommand{\listingcaption}[1]
{%
\vspace*{\abovecaptionskip}\small 
\refstepcounter{lstlisting}\hfill%
Listing \thelstlisting: #1\hfill%\hfill%
\addcontentsline{lol}{lstlisting}{\protect\numberline{\thelstlisting}#1}
}%



% INFO: Pomocnicze marko do wyróżniania tekstu w języku angielskim
\newcommand{\eng}[1]{(ang.~\emph{#1})}
% IFNO: Pomocnicze makro do dołączania podpisów do rysunków ze wskazaniem źródła (bez wypisywania tego źródła w spisie rysunków)
\newcommand*{\captionsource}[2]{%
  \caption[{#1}]{%
    #1 \emph{Źródło:} #2%
  }%
}


% INFO: Makro pozwalające zmienić sposób wypisywania rozdziału (proszę z niego nie korzystać)
%\def\printchaptertitle##1{\fonttitle \space \thechapter.\space ##1} 

% INFO: definicje etykiet i tytułów spisów

%\AtBeginDocument{% 
        \addto\captionspolish{% 
        \renewcommand{\tablename}{Tab.}%% INFO: Przedefiniowanie etykiet w podpisach tabel 
}%} 

%\AtBeginDocument{% 
%        \addto\captionspolish{% 
%        \renewcommand{\chaptername}{Rozdział}% INFO: Przedefiniowanie nazwy rozdziału, niepotrzebne, bo przy polskich ustawieniach językowych jest 'Rozdział'
%}} 

% Przedefiniowanie etykiet oraz nazw wykazu literatury, spisów, indeksu
%\AtBeginDocument{% 
        \addto\captionspolish{% 
        \renewcommand{\figurename}{Rys.}%% INFO: Przedefiniowanie etykiet w podpisach rysunków 
}%}

%\AtBeginDocument{% 
        \addto\captionspolish{% 
        \renewcommand{\lstlistlistingname}{Spis listingów}%% INFO: Przedefiniowanie nazwy spisu listingów
}%} 
\newlistof{lstlistoflistings}{lol}{\lstlistlistingname}


%\AtBeginDocument{% 
        \addto\captionspolish{% 
        \renewcommand{\bibname}{Literatura}%% INFO: Przedefiniowanie nazwy wykazu literatury 
}%}

%\AtBeginDocument{% 
        \addto\captionspolish{% 
        \renewcommand{\listfigurename}{Spis rysunków}%% INFO: Przedefiniowanie nazwy spisu rysunków 
}%}

%\AtBeginDocument{% 
        \addto\captionspolish{% 
        \renewcommand{\listtablename}{Spis tabel}%% INFO: Przedefiniowanie nazwy spisu tabel 
}%}

%\AtBeginDocument{% 
        \addto\captionspolish{% 
\renewcommand\indexname{Indeks rzeczowy}%% INFO: Przedefiniowanie nazwy indeksu 
}%}

%\AtBeginDocument{% 
%    \addto\captionspolish{
%\renewcommand\abstractname{Streszczenie}%% INFO: Przedefiniowanie nazwy strzeszczenia, niepotrzebne, bo przy polskich ustawieniach językowych jest 'Streszczenie'
%}%}

%\AtBeginDocument{% 
%    \addto\captionsenglish{
%\renewcommand\abstractname{Abstract} 
%}%}

\renewcommand{\abstractnamefont}{\normalfont\Large\bfseries}
\renewcommand{\abstracttextfont}{\normalfont}


%%%%%%%%%%%%%%%%%%%%%%%%%%%%%%%%%%%%%%%%%%%%%%%%%%%%%%%%%%%%%%%%%%%%%%%%%%%%%%%%
%% Definicje stopek i nagłówków
%%%%%%%%%%%%%%%%%%%%%%%%%%%%%%%%%%%%%%%%%%%%%%%%%%%%%%%%%%%%%%%%%%%%%%%%%%%%%%%%
\addtopsmarks{headings}{%
\nouppercaseheads % added at the beginning
}{%
\createmark{chapter}{both}{shownumber}{}{. \space}
%\createmark{chapter}{left}{shownumber}{}{. \space}
\createmark{section}{right}{shownumber}{}{. \space}
}%use the new settings

\makeatletter
\copypagestyle{outer}{headings}
\makeoddhead{outer}{}{}{\small\itshape\rightmark}
\makeevenhead{outer}{\small\itshape\leftmark}{}{}
\makeoddfoot{outer}{\small\@author:~\@titleShort}{}{\small\thepage}
\makeevenfoot{outer}{\small\thepage}{}{\small\@author:~\@title}
\makeheadrule{outer}{\linewidth}{\normalrulethickness}
\makefootrule{outer}{\linewidth}{\normalrulethickness}{2pt}
\makeatother

% fix plain
\copypagestyle{plain}{headings} % overwrite plain with outer
\makeoddhead{plain}{}{}{} % remove right header
\makeevenhead{plain}{}{}{} % remove left header
\makeevenfoot{plain}{}{}{}
\makeoddfoot{plain}{}{}{}

\copypagestyle{empty}{headings} % overwrite plain with outer
\makeoddhead{empty}{}{}{} % remove right header
\makeevenhead{empty}{}{}{} % remove left header
\makeevenfoot{empty}{}{}{}
\makeoddfoot{empty}{}{}{}

% INFO: deklaracja zmiennej logicznej wykorzystywanej do rozróżnienia pracy inżynierskiej i magisterskiej
\newif\ifMaster% domyślnie false (czyli domyślnie mamy pracę inżynierską)

%%%%%%%%%%%%%%%%%%%%%%%%%%%%%%%%%%%%%%%%%%%%%%%%%%%%%%%%%%%%%%%%%%%%%%%%%%%%%%%%
%% Definicja strony tytułowej 
%%%%%%%%%%%%%%%%%%%%%%%%%%%%%%%%%%%%%%%%%%%%%%%%%%%%%%%%%%%%%%%%%%%%%%%%%%%%%%%%
\makeatletter
%Uczelnia
\newcommand\uczelnia[1]{\renewcommand\@uczelnia{#1}}
\newcommand\@uczelnia{}
%Wydział
\newcommand\wydzial[1]{\renewcommand\@wydzial{#1}}
\newcommand\@wydzial{}
%Kierunek
\newcommand\kierunek[1]{\renewcommand\@kierunek{#1}}
\newcommand\@kierunek{}
%Specjalność
\newcommand\specjalnosc[1]{\renewcommand\@specjalnosc{#1}}
\newcommand\@specjalnosc{}
%Tytuł po angielsku
\newcommand\titleEN[1]{\renewcommand\@titleEN{#1}}
\newcommand\@titleEN{}
%Tytuł krótki
\newcommand\titleShort[1]{\renewcommand\@titleShort{#1}}
\newcommand\@titleShort{}
%Promotor
\newcommand\promotor[1]{\renewcommand\@promotor{#1}}
\newcommand\@promotor{}
%Słowa kluczowe
\newcommand\kvpl[1]{\renewcommand\@kvpl{#1}}
\newcommand\@kvpl{}
\newcommand\kven[1]{\renewcommand\@kven{#1}}
\newcommand\@kven{}
%Komenda wykorzystywana w streszczeniu
\newcommand\mykeywords{\hspace{\absleftindent}%
\parbox{\linewidth-2.0\absleftindent}{
       \iflanguage{polish}{\textbf{Słowa kluczowe:} \@kvpl}{%
			 \iflanguage{english}{\textbf{Keywords:} \@kven}}{}}
				}

\def\maketitle{%
  \pagestyle{empty}%
%%\garamond 
	\fontfamily{\ebgaramond@family}\selectfont % na stronie tytułowej czcionka garamond
%%%%%%%%%%%%%%%%%%%%%%%%%%%%%%%%%%%%%%%%%%%%%%%%%%%%%%%%%%%%%%%%%%%%%%%%%%%%%%	
%% Poniżej, w otoczniu picture, wstawiono tytuł i autora. 
%% Tytuł (z autorem) musi znaleźć się w obszarze 
%% odpowiadającym okienku 110mmx75mm, którego lewy górny róg 
%% jest w położeniu 77mm od lewej i 111mm od górnej  krawędzi strony 
%% (tak wynika z wycięcia na okładce). 
%% Poniższy kod musi być użyty dokładnie w miejscu gdzie jest.
%% Jeśli tytuł nie mieści się w okienku, to należy tak pozmieniać 
%% parametry użytych komend, aby ten przydługi tytuł jednak 
%% upakować do okienka.
%%
%% Sama okładka (kolorowa strona z wycięciem, kiedyś była do pobrania z dydaktyki) 
%% powinna być przycięta o 3mm od każdej z krawędzi.
%% Te 3mm pewnie zostawiono na ewentualne spady czy też specjalną oprawę.
%%%%%%%%%%%%%%%%%%%%%%%%%%%%%%%%%%%%%%%%%%%%%%%%%%%%%%%%%%%%%%%%%%%%%%%%%%%%%%
\newlength{\tmpfboxrule}
\setlength{\tmpfboxrule}{\fboxrule}
\setlength{\fboxsep}{2mm}
\setlength{\fboxrule}{0mm} 
%\setlength{\fboxrule}{0.1mm} %% INFO: Jeśli chcemy zobaczyć ramkę, wystarczy odmarkować tę linijkę
\setlength{\unitlength}{1mm}
\begin{picture}(0,0)
%\put(26,-124){\fbox{% ustawienie do "wyciętego okienka"
\put(20,-124){\fbox{% ustawienie na środku
\parbox[c][71mm][c]{104mm}{\centering%\lineskip=34pt 
{\fontsize{18pt}{20pt}\bfseries\selectfont \@title}\\[5mm]
{\fontsize{18pt}{20pt}\bfseries\selectfont \@titleEN}\\[10mm] % INFO: wstawiono tytuł w języku angielskim, choć w obecnych oficjalnych zaleceniach tego nie ma
%\fontsize{16pt}{18pt}\selectfont AUTOR:\\[2mm]
{\fontsize{16pt}{18pt}\selectfont \@author}}
}
}
\end{picture}
\setlength{\fboxrule}{\tmpfboxrule} 
%%%%%%%%%%%%%%%%%%%%%%%%%%%%%%%%%%%%%%%%%%%%%%%%%%%%%%%%%%%%%%%%%%%%%%%%%%%%%%
%% Reszta strony z nazwą uczelni, wydziału, kierunkiem, specjalnością
%% promotorem, oceną pracy (zakomentowane), miastem i rokiem
	{\vskip 9pt\centering
		{\fontsize{20pt}{22pt}\bfseries\selectfont \@uczelnia}\\[5pt]
		{\fontsize{16pt}{18pt}\bfseries\selectfont \@wydzial}\\[1pt]
		  \hrule
	}
{\vskip 24pt\raggedright\fontsize{14pt}{16pt}\selectfont%
\begin{tabular}{@{}ll}
Kierunek: & {\bfseries \@kierunek}\\
Specjalność: & {\bfseries \@specjalnosc}\\
\end{tabular}\\[1.3cm]
}
{\vskip 29pt\centering{\fontsize{24pt}{26pt}\selectfont%
{\fontsize{26pt}{28pt}\selectfont P}RACA {\fontsize{26pt}{24pt}\selectfont D}YPLOMOWA\\[7pt]
\ifMaster \selectfont{\fontsize{26pt}{28pt}\selectfont M}AGISTERSKA\\[2.5cm]%
\else \selectfont{\fontsize{26pt}{28pt}\selectfont I}NŻYNIERSKA\\[2.5cm]\fi
}}
	\vfill
{\centering
		{\fontsize{14pt}{16pt}\selectfont Opiekun pracy}\\[2mm] 
		{\fontsize{14pt}{16pt}\bfseries\selectfont \@promotor}\\[10mm]%INFO: tutaj wstawiane ejst nazwisko promotora
%		&{\fontsize{16pt}{18pt}\selectfont OCENA PRACY:}\\[20mm] 
% INFO: linię powyższą zakomentowano, gdyż od czasu pandemii COVID-19 prace mogą być dostarczane bez podpisu promotora
}
\vspace{4cm}\noindent
{\fontsize{12pt}{14pt}\selectfont Słowa kluczowe: \@kvpl}% INFO: na stronę tytułową trafiają tylko słowa kluczowe w języku polskim (w jakim napisana jest praca)
\vspace{1.3cm}
\hrule\vspace*{0.3cm}
{\centering
{\fontsize{14pt}{16pt}\selectfont \@date}\\[0cm]
}
%\ungaramond
\normalfont
 \cleardoublepage
}
\makeatother

%\AtBeginDocument{\addtocontents{toc}{\protect\thispagestyle{empty}}}

%%%%%%%%%%%%%%%%%%%%%%%%%%%%%%%%%%%%%%%%%%%%%%%%%%%%%%%%%%%%%%%%%%%%%%%%%%%%%%%%%%
%%%%%%%%%%%%%%%%%%%%%%%%%%%%%%%%%%%%%%%%%%%%%%%%%%%%%%%%%%%%%%%%%%%%%%%%%%%%%%%%%%
%   Początek strefy do nanoszenia zmian 
%%%%%%%%%%%%%%%%%%%%%%%%%%%%%%%%%%%%%%%%%%%%%%%%%%%%%%%%%%%%%%%%%%%%%%%%%%%%%%%%%%

%%%%%%%%%%%%%%%%%%%%%%%%%%%%%%%%%%%%%%%%%%%%%%%%%%%%%%%%%%%%%%%%%%%%%%%%%%%%%%%%%%
%%%%%%%%%%%%%%%%%%%%%%%%%%%%%%%%%%%%%%%%%%%%%%%%%%%%%%%%%%%%%%%%%%%%%%%%%%%%%%%%%%
%%
%%  Metadane dokumentu
%%  - tutaj należy wstawić własne dane
%%
%%%%%%%%%%%%%%%%%%%%%%%%%%%%%%%%%%%%%%%%%%%%%%%%%%%%%%%%%%%%%%%%%%%%%%%%%%%%%%%%%%

%%%%%%%%%%%%%%%%%%%%%%%%%%%%%%%%%%%%%%%%%%%%%%%%%%%%%%%%%%%%%%%%%%%%%%%%%%%%%%%%%%
%\Mastertrue % INFO: odkomentuj, jeśli to praca magisterska
\title{Narzędzie automatyzujące budowanie projektów w C++} % INFO: tytuł pracy w języku polskim 
\titleShort{Narzędzie budowania projektów}  % INFO: krótki tytuł pracy (do zamieszczenia w stopce, sklejony z imieniem i nazwiskiem autora nie powinien zająć więcej niż jedną linijkę)
\titleEN{	A tool for automating project building in C++} % INFO: tytuł pracy w języku angielskim
\author{Illia Shvarov}  % INFO: imię i nazwisko autora
\uczelnia{Politechnika Wrocławska} % INFO: nazwa uczelni
\wydzial{Wydział Informatyki i Telekomunikacji} % INFO: nazwa wydziału
\kierunek{Informatyka techniczna (ITE)} % IFO: nazwa kierunku
\specjalnosc{Inżynieria systemów informatycznych (INS)} % INFO: nazwa specjalności
\promotor{dr inż. Tomasz Babczyński} % INFO: dane promotora 
\kvpl{C++, cmake, automatyzacja budowania} % INFO: słowa kluczowe po polsku
\kven{C++, cmake, build automation} % INFO: słowa kluczowe po angielsku
\date{WROCŁAW, 2024} % INFO: miejscowość, rok złożenia pracy dyplomowej

%%%%%%%%%%%%%%%%%%%%%%%%%%%%%%%%%%%%%%%%%%%%%%%%%%%%%%%%%%%%%%%%%%%%%%%%%%%%%%%%%%
%%
%%  Struktura dokumentu
%%  - tutaj należy wstawić własne rozdziały
%%
%%%%%%%%%%%%%%%%%%%%%%%%%%%%%%%%%%%%%%%%%%%%%%%%%%%%%%%%%%%%%%%%%%%%%%%%%%%%%%%%%%

%%%%%%%%%%%%%%%%%%%%%%%%%%%%%%%%%%%%%%%%%%%%%%%%%%%%%%%%%%%%%%%%%%%%%%%%%%%%%%%%%%
% INFO: Za pomocą polecenia \includeonly{} można dokonać selekcji  
%       tych części (plików z latexowym kodem), które mają być kompilowane. 
%       Przydaje się to szczególnie podczas pracy nad dużymi dokumentami. 
%       Bo im mniej części zostanie wyselekcjonowanych, tym szybsza będzie kompilacja.
%       Proszę nie mylić tej komendy z poleceniem \include{}, którą używa się 
%       do zadeklarowania pełnej struktury dokumentu (plików z latexowym kodem).
%\includeonly{skroty,rozdzial01}  

\begin{document}
% Komendami poniżej można przełączyć odstęp między liniami. Proszę jednak tego nie robić !!!
%\SingleSpacing
%\OnehalfSpacing
%\DoubleSpacing

%\settypeoutlayoutunit{cm} % do debugowania
%\typeoutstandardlayout    % wypisuje na stdout informacje o ustawieniach

%\frontmatter
\pdfbookmark[0]{Tytuł}{Tytul.1}
\maketitle

% Kolejne części dokumentu: streszczenie, spisy, skróty, rozdziały, dodatki
%\chapterstyle{noNumbered}
% STRESZCZENIE (proszę zajrzeć do środka na zakomentowane komendy)
%\pdfbookmark[0]{Streszczenie}{streszczenie.1}
%\phantomsection
%\addcontentsline{toc}{chapter}{Streszczenie}
%%% Poniższe zostało niewykorzystane (tj. zrezygnowano z utworzenia nienumerowanego rozdziału na abstrakt)
%%%\begingroup
%%%\setlength\beforechapskip{48pt} % z jakiegoś powodu była maleńka różnica w położeniu nagłówka rozdziału numerowanego i nienumerowanego
%%%\chapter*{\centering Abstrakt}
%%%\endgroup
%%%\label{sec:abstrakt}
%%%Lorem ipsum dolor sit amet eleifend et, congue arcu. Morbi tellus sit amet, massa. Vivamus est id risus. Sed sit amet, libero. Aenean ac ipsum. Mauris vel lectus. 
%%%
%%%Nam id nulla a adipiscing tortor, dictum ut, lobortis urna. Donec non dui. Cras tempus orci ipsum, molestie quis, lacinia varius nunc, rhoncus purus, consectetuer congue risus. 
%\mbox{}\vspace{2cm} % można przesunąć, w zależności od długości streszczenia
\begin{abstract}
    Streszczenie w języku polskim powinno zmieścić się na połowie strony. Drugą połowę powinno zająć streszczenie w języku angielskim. Zwykle w streszczeniu krótko nawiązuje się do tematu pracy, potem przybliża zawartość pracy oraz osiągnięte wyniki. Czasem w streszczeniu zamieszczana jest notka o wykorzystanym stosie technologii.
    Styl wypowiedzi powinien być odpowiedni (język formalny, styl akademicki). Należy pamiętać, że w języku polskim i angielskim obowiązują inne zasady. Pomijając użycie czasów do określenia czynności wykonywanych lub planowanych zazwyczaj polski tekst naukowy redagowany jest w trybie dokonanym, w~formie bezosobowej. Natomiast zasady dotyczące manuskryptów w języku angielskim zakładają: głos czynny, forma osobowa. Należy pamiętać, że treść obu abstraktów prawdopodobnie nie będą zgadzać się 1:1.

    W streszczeniu nie stosuje się zaawansowanego formatowania (list wyliczeniowych, wyróżnień, tabel itp.). Można jednak je zredagować w formie kilku (dwóch, może trzech) akapitów.  Pierwszy może przybliżać temat pracy, drugi może dotyczyć przebiegu pracy i jej zawartości. Oczywiście redagowanie pracy jest procesem twórczym i trudno tu narzucać jakieś sztywne reguły. Ujmując rzecz ogólnie, streszczenie powinno umożliwić czytelnikowi zapoznanie się z istotą i zawartością pracy.
\end{abstract}
\mykeywords

{
    \selectlanguage{english}
    \begin{abstract}
        The abstract in Polish should fit on half a page. The other half should be taken up by an abstract in English. Usually the abstract briefly refers to the thesis topic, then introduces the scope of the work and the results achieved. Sometimes the abstract includes a note about the technology stack used. The writing style should be appropriate (formal language, academic style). Note that Polish and English have different rules. Leaving aside the use of tenses indicating actions performed or planned, the Polish scientific text is generally edited in the accomplished mode and in impersonal form. The rules for English manuscripts are: active voice, personal form. Note that the content of the two abstracts will probably not match 1:1.

        Any advanced formatting (enumeration lists, highlights, tables, etc.) is not allowed here. But the abstract might consist of a couple of paragraphs (two, maybe three). The first can introduce the topic of the paper; the second can be about the course of the work and the thesis' content. Of course, writing is a creative process, and it is difficult to impose any rigid rules here. Generally speaking, the abstract should allow the reader to get an idea of the essence and content of the thesis.
    \end{abstract}
    \mykeywords
}

\pagestyle{outer}
\clearpage
% SPIS TREŚCI (zostanie wygenerowany automatycznie)
\pdfbookmark[0]{Spis treści}{spisTresci.1}%
%%\phantomsection
%%\addcontentsline{toc}{chapter}{Spis treści}
\tableofcontents*
\clearpage
% SPIS RYSUNKÓW (zostanie wygenerowany automatycznie)
\pdfbookmark[0]{Spis rysunków}{spisRysunkow.1} % jeśli chcemy mieć w spisie treści, to zamarkować tę linię, a odmarkować linie poniższe
%%\phantomsection
%%\addcontentsline{toc}{chapter}{Spis rysunków}
\listoffigures*
\clearpage
% SPIS TABEL (zostanie wygenerowany automatycznie)
\pdfbookmark[0]{Spis tabel}{spisTabel.1} %
%%\phantomsection
%%\addcontentsline{toc}{chapter}{Spis tabel}
\listoftables*
\clearpage
% SPIS LISTINGÓW (zostanie wygenerowany automatycznie)
\pdfbookmark[0]{Spis listingów}{spisListingow.1} %
%%\phantomsection
%%\addcontentsline{toc}{chapter}{Spis listingów}
\lstlistoflistings*
\clearpage
% SKRÓTY (to opcjonalna część pracy)
\pdfbookmark[0]{Skróty}{skroty.1}% 
%%\phantomsection
%%\addcontentsline{toc}{chapter}{Skróty}
\chapter*{Skróty}
\label{sec:skroty}
\noindent\vspace{-\topsep-\partopsep-\parsep} % Jeśli zaczyna się od otoczenia description, to otoczenie to ląduje lekko niżej niż wylądowałby zwykły tekst, dlatego wstawiano przesunięcie w pionie
\begin{description}[labelwidth=*]
  \item [BBS] (ang.\ \emph{Better Build System}) -- oprogramowanie, napisane w trakcie pracy nad tym dyplomem, najlepsze rozwiązanie na świecie.
  \item [CI] (ang.\ \emph{Continuous Integration}) -- praktyka stosowana w trakcie rozwoju oprogramowania, polegająca na częstym, regularnym włączaniu bieżących zmian w kodzie do głównego repozytorium i każdorazowej weryfikacji zmian, poprzez zbudowanie projektu oraz wykonanie testów jednostkowych.
  \item [GCC] (ang.\ \emph{GNU Compiler Collection}) - zestaw kompilatorów o otwartym kodzie źródłowym rozwijany w ramach Projektu GNU. 
  \item [GNU] (ang.\ \emph{"GNU's Not Unix!"}) - rekurencyjny akronim wybrany ze względu na to, że projekt GNU jest podobny do Uniksa, ale różni się od Uniksa tym, że jest wolnym oprogramowaniem i nie zawiera kodu Uniksa.
  \item [HTML] (ang.\ \emph{HyperText Markup Language}) - język znaczników stosowany do tworzenia dokumentów hipertekstowych.
\end{description}

% ROZDZIAŁY (kolejne rozdziały dołączane są z kolejnych plików)
\chapterstyle{default}
\chapter{Wstęp}
\section{Opis projektu}

Projekt nazywa się BBS -- to jest skrót od ang. \textbf{B}etter \textbf{B}uild \textbf{S}ystem. Dalej w tym dokumencie będę wykorzystywać skrót w celu odwołania do swojego projektu.

BBS to narzędzie, które automatyzuje proces budowania projektów napisanych w języku C++. W tradycyjnym, ręcznym podejściu do budowania dużych projektów C++, konieczność ręcznej konfiguracji i kompilacji wielu plików źródłowych często prowadzi do błędów i jest bardzo czasochłonna. To nie tylko wydłuża czas potrzebny na testowanie, ale może również wprowadzać nieprzewidywalne i trudne do wykrycia błędy. Celem BBS jest uproszczenie oraz usprawnienie procesu kompilacji, linkowania i zarządzania zależnościami między plikami źródłowymi, co jest szczególnie istotne w przypadku skomplikowanych projektów o dużej ilości kodu.

\subsection{Cechy}
Projekt, żeby być naprawdę użytecznym, musi posiadać następujące cechy, które będą wyróżniały go spośród innych podobnych narzędzi:

\begin{itemize}
    \item konfigurowalność z pliku: zarządzanie procesem budowania musi być wykonane za pomocą pliku konfiguracyjnego, żeby uniezależnić system od wybranego środowiska programowania lub systemu ciągłej integracji (CI),
    \item prosta składnia języka: język, który będzie wymyślony specjalnie do wykorzystania w plikach konfiguracyjnych projektu, musi posiadać jak najmniej operacji i symboli, a także mieć minimum symboli, wykorzystanych do separacji kolejnych elementów pliku (przykładem jest ,,;'' w języku C++),
    \item minimum słów kluczowych: język musi posiadać tylko i wyłącznie słowa niezbędne do opisu projektu, który należy zbudować, ponadto każde słowo musi pełnić tylko i wyłącznie jedną rolę,
    \item przejrzystość i jednoznaczność: program, który będzie napisany w celu interpretacji wymyślonego języka plików konfiguracyjnych, musi nie mieć żadnych ukrytych parametrów czy zmiennych.
\end{itemize}

\subsection{Główne funkcje}
Dla kompletnego opisu projektu także muszą być podane jego główne funkcje. Oto lista wszystkich funkcji BBS:

\begin{itemize}
    \item eliminacja zbędnych operacji: system automatycznie śledzi zmiany w plikach źródłowych i unika zbędnych operacji kompilacji plików obiektowych czy linkowania przy kolejnym odpaleniu, co znacznie skraca czas budowania projektu,
    \item wieloplatformowość: Obsługuje główne systemy operacyjne, takie jak Linux i Windows,
    \item elastyczne Konfiguracje: Możliwość dostosowywania ustawień kompilacji za pomocą prostych skryptów konfiguracyjnych i wsparcie dla niestandardowych flag kompilatora.
\end{itemize}

\section{Inne rozwiązania}
Język C++ standardowo nie ma swojego systemu do budowania, jak to jest np. w języku Rust \cite{rust-cargo}. Skoro projekty w tym języku stawały się i nadal stają coraz większe i bardziej skomplikowane, musiały powstać takie narzędzia. Najpopularniejszymi teraz są Make, CMake, Visual Studio(VS) i Meson. Każdy z nich będzie opisany poniżej.

\subsection{Make}
Make\cite{make} jest jednym z najstarszych i najbardziej podstawowych narzędzi do budowania projektów, które polega na ręcznym tworzeniu plików Makefile. To narzędzie jest stosunkowo niskopoziomowe i wymaga, aby użytkownik samodzielnie zdefiniował wszystkie reguły budowania oraz zależności.

\subsubsection{Zalety}
Make to bardzo elastyczne narzędzie, które umożliwia pełną kontrolę nad procesem budowania. Ono jest dobrze znane i jest prawie zawsze wykorzystane do budowania projektów w systemach pochodnych od UNIX (bezpośrednio lub pośrednio przez CMake \cite{cmake}). 

Także istnieje podobne narzędzie od Microsoft, które nazywa się \textbf{NMake}, które również działa na podstawie plików makefile, ale jest ściśle związane z kompilatorami z Visual Studio i oferuje ograniczoną funkcjonalność w porównaniu do Make, a także ma zupełnie inną składnię \cite{nmake}.

\subsubsection{Wady}
Ale także Make ma swoje wady: ręczne zarządzanie zależnościami jest podatne na błędy, może być czasochłonne i trudne w utrzymaniu w dużych projektach. Także to rozwiązanie nie jest wieloplatformowym, ponieważ nie ma bezpośredniego wsparcia tego systemu w Windows, gdzie najczęściej są wykorzystane inne narzędzia.

\subsection{CMake}
CMake to najpopularniejsze narzędzie do budowania dla C++, które generuje skrypty budowania (np. Makefile, projekty Visual Studio) na podstawie prostych plików konfiguracyjnych \cite{cmake}. CMake skupia się na wieloplatformowości i szerokiej kompatybilności z różnymi narzędziami budowania.

\subsubsection{Zalety}
Zalety systemu budowania projektów CMake są naprawdę znaczące, co pozwoliło temu narzędziu stać się jednym z najpopularniejszych systemów. Oto one:

\begin{itemize}
    \item szerokie wsparcie dla różnych platform i kompilatorów,
    \item duża elastyczność w zarządzaniu projektami,
    \item automatyczne generowanie skryptów budowania dla różnych środowisk,
    \item CMake jest projektem z otwartym kodem źródłowym.
\end{itemize}

\subsubsection{Wady}
Niestety, CMake nie jest idealnym rozwiązaniem dla wszystkich użytkowników, więc ma ono swoje wady:

\begin{itemize}
    \item pliki CMakeLists.txt mogą być skomplikowane w dużych projektach,
    \item nowi użytkownicy tracą dużo czasu na nauczenie się korzystania z tego narzędzia,
    \item konfiguracja może być czasochłonna w przypadku bardziej zaawansowanych scenariuszy,
    \item kod źródłowy CMake jest bardzo trudny do przejrzenia i ma skomplikowaną strukturę.
\end{itemize}

\subsection{Visual Studio}
Visual Studio (VS) to rozbudowane środowisko IDE z wbudowanym systemem budowania, które jest często używane w projektach C++ na platformach Windows \cite{vs}. Narzędzie korzysta z plików projektowych .vcxproj i dostarcza zintegrowane funkcje do debugowania, analizy i zarządzania kodem \cite{vs-cpp-debug}.

\subsubsection{Zalety}
Visual Studio jest kompleksowym IDE z integracją z narzędziami do debugowania, analizy i testowania kodu. To znaczy, iż użytkownicy mają rozwiązanie ,,wszystko w jednym'' dla programowania w C++ z bardzo przyjaznym interfejsem i dużą ilością rozszerzeń do napisania wydajnego kodu.

\subsubsection{Wady}
Niestety, skoro to rozwiązanie jest napisane przez Microsoft, nie stosuje się go na innych systemach operacyjnych, niż Windows. Także strategia wykorzystania VS automatycznie eliminuje możliwość użycia własnych edytorów tekstu czy narzędzi CI.

\subsection{Meson}
Meson to nowoczesny system budowania, który kładzie nacisk na szybkość i łatwość konfiguracji \cite{meson}. Meson używa prostych plików konfiguracyjnych meson.build i współpracuje z narzędziem Ninja, które przyspiesza proces kompilacji.

\subsubsection{Zalety}
Meson ma bardzo szybki czas budowania dzięki integracji z Ninja, z czego także wynika wieloplatformowość i wsparcie dla szerokiego zakresu kompilatorów. Także warto dodać, iż  pliki konfiguracyjne \texttt{meson.build} mają prostą strukturę i nie wymagają dużo czasu do zrozumienia \cite{meson-sample}.

\subsubsection{Wady}
Meson nie pracuje z kompilatorami języków programowania (np. GNU GCC), czyli nie jest osobnym narzędziem dla budowania, raczej kompilatorem z własnego (i bardzo wygodnego) języka do języka, zrozumianego przez Ninja.

\section{Cel i zakres projektu}
Celem mojej pracy inżynierskiej jest stworzenie narzędzia automatyzującego proces budowania projektów w C++, które uprości zarządzanie złożonymi zależnościami. Narzędzie będzie korzystać z prostej gramatyki i ograniczonej liczby zmiennych, aby ułatwić użytkowanie, a także wykorzysta nowoczesne standardy języka C++ dla zapewnienia przejrzystości i modułowości kodu. Dzięki temu narzędzie zwiększy efektywność i niezawodność budowania projektów na różnych platformach.

Zakres pracy wynika bezpośrednio z celów mojej pracy, opisanych powyżej, i obejmuje niżej wymienione elementy.
\begin{itemize}
    \item Analizę istniejących narzędzi do automatyzacji budowania projektów w C++, takich jak CMake, Make, autoconf, meson oraz Visual Studio, z uwzględnieniem ich zalet i ograniczeń.
    \item Opracowanie wymagań dla nowego narzędzia, które zapewni uproszczony proces budowania z prostą gramatyką.
    \item Implementację narzędzia z użyciem nowoczesnych standardów C++, z naciskiem na przejrzystość kodu i modularność.
    \item Testowanie narzędzia w projektach o różnej złożoności i napisanie testów jednostkowych (unit tests).
    \item Opracowanie dokumentacji użytkownika (za pomocą Doxygen) oraz przykładów użycia narzędzia.
\end{itemize}
% LITERATURA (zostanie wygenerowana automatycznie)
%UWAGA: bibliotekę referencji należy przygotować samemu. Dobrym do tego narzędziem jest JabRef.
%       JabRef oferuje jednak większą liczbę typów rekordów niż obsługuje BibTeX.
%       Proszę nie deklarować rekordów o typach nieobsługiwanych przez BibTeX.
%       Formatowania wykazu literatury i cytowań odbywać się ma zgodnie z zadeklarowanym stylem.
%       Zalecane są style produkujące numeryczne cytowania (w postaci [1], [2,3]).
%       Takim stylem jest np. plabbrv
\bibliographystyle{plabbrv}
%       Aby zapanować nad odstępami w wykazie literatury można posłużyć się poniższą komendą
\setlength{\bibitemsep}{2pt} % - zacieśnia wykaz
%       Pozycja Literatura pojawia się w spisie treści nieco inaczej niż spisy rysunków, tabel itp.
%       Aby zachować właściwe odstępy należy użyć poniższej komendy
\addtocontents{toc}{\addvspace{2pt}} % ustawiamy odstęp w spisie treści przed pozycją Literatura 
%       Nazwę pliku przygotowanej biblioteki wpisuje się bez rozszerzenia .bib
%       (linia poniżej załaduje rekordy z pliku "dokumentacja.bib")
\bibliography{dokumentacja}
\appendix
%\chapter{Instrukcja wdrożeniowa}
Jeśli praca skończyła się wykonaniem jakiegoś oprogramowania, to w dodatku powinna pojawić się instrukcja wdrożeniowa (o tym jak skompilować/zainstalować to oprogramowanie).
Przydałoby się również krótkie ,,\emph{how to}'' (jak uruchomić system i coś w nim zrobić -- zademonstrowane na jakimś najprostszym przypadku użycia). Można z tego zrobić osobny dodatek.
%\chapter{Instrukcja Obsługi}

Poniżej znajduje się instrukcja uruchomienia aplikacji BBS oraz opis formatu pliku konfiguracyjnego wykorzystywanego do definiowania procesu budowy.

\section{Proces uruchomienia}

Aby uruchomić aplikację BBS, należy wykonać następujące kroki:

\begin{enumerate}
	\item Przygotuj plik konfiguracyjny w formacie tekstowym, opisujący szczegóły projektu, takie jak nazwa projektu, pliki źródłowe, biblioteki zależne, flagi kompilatora itp.
	\item W terminalu przejdź do katalogu, w którym będzie się znajdował zbudowany projekt.
	\item Uruchom aplikację, podając jako argument ścieżkę do pliku konfiguracyjnego. Przykładowa komenda:
	\begin{lstlisting}
	./bbs ..
	\end{lstlisting}
	\item Aplikacja rozpocznie proces budowy projektu zgodnie z definicjami zawartymi w pliku konfiguracyjnym. W terminalu będą wyświetlane informacje o postępie.
\end{enumerate}
\section{Format pliku konfiguracyjnego}

Plik konfiguracyjny w BBS jest oparty na prostym formacie tekstowym, w którym definiowane są wszystkie aspekty projektu. Poniżej znajduje się szczegółowy opis słów kluczowych używanych w tym pliku:

\paragraph{\texttt{!let}} Pozwala na definiowanie zmiennych, które mogą być wykorzystywane w innych sekcjach pliku. Zmienna musi mieć unikalną nazwę oraz wartość w formie ciągu znaków.
\begin{lstlisting}[language=sh,alsoletter={!},keywords={!let,!prj,!files,!deps,!cflags,!pre,!post,!inc},keywordstyle=\bfseries]
!let project_name = "MyApp"
\end{lstlisting}

\paragraph{\texttt{!prj}} Definiuje nazwę projektu. Może zawierać bezpośrednią wartość lub odwołanie do zmiennej.
\begin{lstlisting}[language=sh,alsoletter={!},keywords={!let,!prj,!files,!deps,!cflags,!pre,!post,!inc}]
!prj "MyApp"
!prj $project_name
\end{lstlisting}

\paragraph{\texttt{!files}} Określa listę plików źródłowych do skompilowania. Pliki są umieszczone w nawiasach kwadratowych, oddzielone przecinkami i ujęte w cudzysłowy.
\begin{lstlisting}[language=sh,alsoletter={!},keywords={!let,!prj,!files,!deps,!cflags,!pre,!post,!inc}]
!files [ "main.cpp", "utils.cpp", "app.cpp" ]
\end{lstlisting}

\paragraph{\texttt{!deps}} Definiuje projekty zależne, które powinny zostać zbudowane przed głównym projektem. Musi być podana względna ścieżka do tych projektów.
\begin{lstlisting}[language=sh,alsoletter={!},keywords={!let,!prj,!files,!deps,!cflags,!pre,!post,!inc}]
!deps [ "dependency" ]
\end{lstlisting}

\paragraph{\texttt{!cflags}} Określa flagi kompilatora, takie jak opcje optymalizacji, ostrzeżenia czy inne parametry.
\begin{lstlisting}[language=sh,alsoletter={!},keywords={!let,!prj,!files,!deps,!cflags,!pre,!post,!inc}]
!cflags "-O2 -Wall -lsomelib"
\end{lstlisting}

\paragraph{\texttt{!pre}} Definiuje polecenia do wykonania przed rozpoczęciem procesu kompilacji, np. wyświetlenie komunikatu w terminalu.
\begin{lstlisting}[language=sh,alsoletter={!},keywords={!let,!prj,!files,!deps,!cflags,!pre,!post,!inc}]
!pre "echo 'Rozpoczynam budowanie projektu...'"
\end{lstlisting}

\paragraph{\texttt{!post}} Definiuje polecenia wykonywane po zakończeniu procesu kompilacji, np. informowanie o sukcesie lub kopiowanie plików.
\begin{lstlisting}[language=sh,alsoletter={!},keywords={!let,!prj,!files,!deps,!cflags,!pre,!post,!inc}]
!post "echo 'Budowanie zakończone!'"
\end{lstlisting}

\paragraph{\texttt{!inc}} Określa ścieżki do katalogów z plikami nagłówkowymi wymaganymi w procesie kompilacji.
\begin{lstlisting}[language=sh,alsoletter={!},keywords={!let,!prj,!files,!deps,!cflags,!pre,!post,!inc}]
!inc [ "include/", "external/includes/" ]
\end{lstlisting}

Każde z tych słów kluczowych jest integralną częścią procesu budowy projektu w BBS, umożliwiając elastyczną kontrolę nad kompilacją i konfiguracją projektu. Przygotowując plik konfiguracyjny zgodnie z opisanymi wytycznymi, można w pełni wykorzystać możliwości tego narzędzia.

\section{Działanie aplikacji}
Po uruchomieniu aplikacja odczytuje plik konfiguracyjny i realizuje zdefiniowane w nim kroki. Proces ten obejmuje:

\begin{itemize}
    \item Sprawdzenie, czy wszystkie pliki źródłowe i katalogi istnieją.
    \item Przygotowanie flag i parametrów kompilatora.
    \item Kompilację plików źródłowych i ich linkowanie do postaci programu wykonywalnego.
    \item Wykonanie poleceń określonych w sekcjach pre i post, jeśli zostały podane.
\end{itemize}

Jeśli podczas budowania projektu wystąpią błędy, takie jak brak plików źródłowych lub nieprawidłowa składnia w pliku konfiguracyjnym, aplikacja zgłasza je w formie komunikatów błędów w terminalu.

% Jeśli w pracy pojawiać się ma indeks, należy odkomentować poniższe linie
%%\chapterstyle{noNumbered}
%%\phantomsection % sets an anchor
%%\addcontentsline{toc}{chapter}{Indeks rzeczowy}
%%\printindex

\end{document}
