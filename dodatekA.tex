\chapter{Instrukcja wdrożeniowa}

\section{Wprowadzenie}
Better Build System (BBS) to dedykowane narzędzie automatyzujące proces budowania projektów w języku C++. Jego celem jest uproszczenie kompilacji, linkowania oraz zarządzania zależnościami w rozbudowanych projektach, oferując spójny i wydajny sposób pracy. Dzięki wsparciu społeczności i naciskowi na wydajność oraz elastyczność, BBS umożliwia efektywne zarządzanie dużymi bazami kodu w języku C++.

\section{Wymagania wstępne}
Aby zainstalować BBS, upewnij się, że Twój system spełnia poniższe wymagania:
\begin{itemize}
    \item System operacyjny: macOS, Linux, FreeBSD, OpenBSD lub inne systemy UNIX-podobne (mogą wymagać drobnych modyfikacji),
    \item Narzędzia: zainstalowany CMake, kompilator zgodny z C++17 (np. GCC 14), Doxygen (opcjonalnie, do generowania dokumentacji), GTest(opcjonalnie, służy do uruchomienia testów jednostkowych).
\end{itemize}

\section{Proces instalacji}
Proces instalacji Better Build System (BBS) jest prosty i przejrzysty. Rozpoczyna się od pobrania kodu źródłowego z repozytorium projektu. Następnie należy utworzyć katalog \texttt{build} i w nim uruchomić CMake, aby wygenerować pliki konfiguracyjne odpowiednie dla wybranego systemu i generatora. Po tym kroku wystarczy skompilować projekt poleceniem \texttt{make}, a opcjonalnie można go zainstalować w systemie za pomocą \texttt{make install}. Dodatkowo istnieje możliwość wygenerowania dokumentacji za pomocą Doxygena oraz przeprowadzenia testów jednostkowych przy użyciu \texttt{ctest}, co pozwala zweryfikować poprawność działania narzędzia.

\paragraph{Pobranie kodu źródłowego:} 
Skopiuj repozytorium projektu BBS z wybranego źródła (np. GitHub lub inna platforma hostingowa).

\paragraph{Budowanie projektu:}
\begin{enumerate}
    \item Otwórz terminal i przejdź do katalogu głównego projektu:
    \begin{verbatim}
    cd /ścieżka/do/projektu
    \end{verbatim}
    \item Utwórz katalog dla plików budowania:
    \begin{verbatim}
    mkdir build && cd build
    \end{verbatim}
    \item Uruchom CMake w celu wygenerowania plików konfiguracyjnych:
    \begin{verbatim}
    cmake ..
    \end{verbatim}
\end{enumerate}

\paragraph{Kompilacja:} 
W katalogu \texttt{build} uruchom proces budowania:
\begin{verbatim}
make
\end{verbatim}

\paragraph{Instalacja:} 
Zainstaluj BBS w systemie (opcjonalnie, wymaga uprawnień administratora albo wykorzystania \texttt{sudo}):
\begin{verbatim}
make install
\end{verbatim}

\section{Generowanie dokumentacji}
Jeśli chcesz wygenerować dokumentację:
\begin{enumerate}
    \item Zainstaluj Doxygen (jeśli jeszcze tego nie zrobiłeś).
    \item W katalogu głównym projektu uruchom Doxygen z dostarczonym plikiem konfiguracyjnym:
    \begin{verbatim}
    doxygen Doxyfile
    \end{verbatim}
\end{enumerate}
Wygenerowana dokumentacja będzie dostępna w katalogu \texttt{docs}.

\section{Testowanie}
Aby uruchomić testy jednostkowe projektu:
\begin{enumerate}
    \item Uruchom CMake z poniższymi flagami oraz Make w celu budowania projektu oraz testów:
    \begin{verbatim}
    cmake -DCMAKE_BUILD_TYPE=Debug .. && make
    \end{verbatim}
    \item Po kompilacji w katalogu \texttt{build/tests} uruchom polecenie:
    \begin{verbatim}
    ctest
    \end{verbatim}
\end{enumerate}
Wyniki testów pozwolą zweryfikować poprawność działania narzędzia.