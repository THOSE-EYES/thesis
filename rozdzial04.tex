\chapter{Podsumowanie}
Praca nad projektem BBS umożliwiła stworzenie narzędzia, które usprawnia proces budowania projektów w języku C++. Automatyzacja kompilacji, linkowania oraz zarządzania zależnościami pozwala znacząco uprościć pracę programistów, szczególnie w przypadku dużych i~złożonych projektów.

Analiza dostępnych narzędzi, takich jak \textbf{Make}, \textbf{CMake}, \textbf{NMake} czy \textbf{Meson}, wykazała, że każde z nich ma swoje mocne strony, jednak mogą być one trudne w użyciu lub nadmiarowe w~stosunku do wymagań mniejszych projektów. BBS celuje w dostarczenie intuicyjnego i wydajnego rozwiązania, które eliminuje te problemy.

Wnioski z pracy nad tym projektem obejmują:
\begin{itemize}
    \item Znaczenie automatyzacji: Automatyzacja procesu budowania, szczególnie w przypadku dużych projektów, pozwala uniknąć błędów, oszczędza czas i zwiększa produktywność zespołów programistycznych.
    \item Znaczenie czytelności kodu i dokumentacji: Korzystanie z takich narzędzi jak \textit{Doxygen} i~clang-format wspiera utrzymanie standardów kodu, co ułatwia pracę nad projektem zarówno obecnym, jak i przyszłym członkom zespołu.
    \item Rozwijanie standardów w C++: Nowoczesne cechy języka, takie jak std::filesystem czy std::optional, znacząco upraszczają implementację i poprawiają czytelność kodu.
\end{itemize}

Projekt BBS pokazuje, że możliwe jest stworzenie narzędzia do budowania projektów w C++, które jest lekkie, wydajne i łatwe w użyciu. Dalsze kierunki rozwoju obejmują dodanie wsparcia dla bardziej zaawansowanych funkcji, takich jak dużo lepsze raportowanie błędów z pięknym formatowaniem, lepsze zarządzanie wątkami i jednoczesne budowanie kilku projektów, rozszerzenie listy wspieranych języków i kompilatorów oraz integracja z istniejącymi IDE (np. Visual Studio).

Realizacja tego projektu pozwoliła na zdobycie praktycznych umiejętności z zakresu nowoczesnego C++, inżynierii oprogramowania (w szczególności kompilatorów i interpreterów) oraz pracy z narzędziami wspierającymi rozwój dużych projektów. Wyniki pracy mogą być z powodzeniem wykorzystane jako fundament dla przyszłych projektów czy w celu zastąpienia innych rozwiązań nowoczesnym, wydajnym i łatwym w obsłudze oprogramowaniem.