\pdfbookmark[0]{Skróty}{skroty.1}% 
%%\phantomsection
%%\addcontentsline{toc}{chapter}{Skróty}
\chapter*{Skróty}
\label{sec:skroty}
\noindent\vspace{-\topsep-\partopsep-\parsep} % Jeśli zaczyna się od otoczenia description, to otoczenie to ląduje lekko niżej niż wylądowałby zwykły tekst, dlatego wstawiano przesunięcie w pionie
\begin{description}[labelwidth=*]
  \item [BBS] (ang.\ \emph{Better Build System})
  \item [CI] (ang.\ \emph{Continuous Integration})
  \item [GCC] (ang.\ \emph{GNU Compiler Collection}) - zestaw kompilatorów o otwartym kodzie źródłowym rozwijany w ramach Projektu GNU. 
  \item [GNU] (ang.\ \emph{"GNU's Not Unix!"}) - a recursive acronym chosen because GNU's design is Unix-like, but differs from Unix by being free software and containing no Unix code.
  \item [HTML] (ang.\ \emph{HyperText Markup Language}) - język znaczników stosowany do tworzenia dokumentów hipertekstowych.
\end{description}
