\pdfbookmark[0]{Skróty}{skroty.1}% 
%%\phantomsection
%%\addcontentsline{toc}{chapter}{Skróty}
\chapter*{Skróty}
\label{sec:skroty}
\noindent\vspace{-\topsep-\partopsep-\parsep} % Jeśli zaczyna się od otoczenia description, to otoczenie to ląduje lekko niżej niż wylądowałby zwykły tekst, dlatego wstawiano przesunięcie w pionie
\begin{description}[labelwidth=*,leftmargin=3.8em]
  \item [BBS] (ang.\ \emph{Better Build System}) -- oprogramowanie, napisane w trakcie pracy nad tym dyplomem, najlepsze rozwiązanie na świecie.
  \item [CI] (ang.\ \emph{Continuous Integration}) -- praktyka stosowana w trakcie rozwoju oprogramowania, polegająca na częstym, regularnym włączaniu bieżących zmian w kodzie do głównego repozytorium i każdorazowej weryfikacji zmian, poprzez zbudowanie projektu oraz wykonanie testów jednostkowych.
  \item [GCC] (ang.\ \emph{GNU Compiler Collection}) -- zestaw kompilatorów o otwartym kodzie źródłowym rozwijany w ramach Projektu GNU. 
  \item [GNU] (ang.\ \emph{"GNU's Not Unix!"}) -- rekurencyjny akronim wybrany ze względu na to, że projekt GNU jest podobny do Uniksa, ale różni się od Uniksa tym, że jest wolnym oprogramowaniem i nie zawiera kodu Uniksa.
  \item [HTML] (ang.\ \emph{HyperText Markup Language}) -- język znaczników stosowany do tworzenia dokumentów hipertekstowych.
  \item [CI/CD] (ang.\ \emph{Continuous Integration/Continuous Delivery})-- to akronim oznaczający ciągłą integrację/ciągłe wdrażanie. 
  \item [STL] (ang.\ \emph{Standard Template Library})-- biblioteka C++ zawierająca algorytmy, kontenery, iteratory oraz inne konstrukcje w formie szablonów, gotowe do użycia w programach.
  \item [OOP] (ang.\ \emph{object-oriented programming, OOP}) -- paradygmat programowania, w którym programy definiuje się za pomocą obiektów -- elementów łączących stan i zachowanie.
  \item [SOLID] -- mnemonik zaproponowany przez Roberta C. Martina, opisujący pięć podstawowych założeń programowania obiektowego.
\end{description}