\chapter{Implementacja}
\section{Wprowadzenie}
Celem tego rozdziału jest szczegółowy opis podejścia do realizacji projektu BBS, a mianowicie wyjaśnienie podstawowych pojęć, analiza architektury oraz algorytmów stosowanych do przetwarzania danych.

Chciałbym zacząć od tego, czym właściwie jest BBS. Zgodnie z definicją projekt jest w rzeczywistości procesorem języka [cytat], a konkretnie tłumaczem [cytat] według następujących cech, które wynikają bezpośrednio z wymagań opisanych w sekcji 2:
\begin{itemize}
    \item Oprogramowanie analizuje kod wejściowy, napisany w specjalnie opracowanym na potrzeby projektu języku
    \item Oprogramowanie powinno wykrywać i raportować możliwe błędy logiczne lub składniowe w pliku wejściowym
    \item Zamiast tłumaczyć plik wejściowy na inny język programowania, program powinien wykonać polecenia opisane w pliku wejściowym
\end{itemize}

Dodatkowo oprogramowanie to zawiera również preprocesor. Dzieje się tak, aby plik wejściowy mógł zostać podzielony na wiele części, co umożliwi działanie mechanizmu budowania podprojektu.