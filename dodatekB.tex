\chapter{Instrukcja Obsługi}

Poniżej znajduje się instrukcja uruchomienia aplikacji BBS oraz opis formatu pliku konfiguracyjnego wykorzystywanego do definiowania procesu budowy.

\section{Proces uruchomienia}

Aby uruchomić aplikację BBS, należy wykonać następujące kroki:

\begin{enumerate}
	\item Przygotuj plik konfiguracyjny w formacie tekstowym, opisujący szczegóły projektu, takie jak nazwa projektu, pliki źródłowe, biblioteki zależne, flagi kompilatora itp.
	\item W terminalu przejdź do katalogu, w którym będzie się znajdował zbudowany projekt.
	\item Uruchom aplikację, podając jako argument ścieżkę do pliku konfiguracyjnego. Przykładowa komenda:
	\begin{lstlisting}
	./bbs ..
	\end{lstlisting}
	\item Aplikacja rozpocznie proces budowy projektu zgodnie z definicjami zawartymi w pliku konfiguracyjnym. W terminalu będą wyświetlane informacje o postępie.
\end{enumerate}
\section{Format pliku konfiguracyjnego}

Plik konfiguracyjny w BBS jest oparty na prostym formacie tekstowym, w którym definiowane są wszystkie aspekty projektu. Poniżej znajduje się szczegółowy opis słów kluczowych używanych w tym pliku:

\paragraph{\texttt{!let}} Pozwala na definiowanie zmiennych, które mogą być wykorzystywane w innych sekcjach pliku. Zmienna musi mieć unikalną nazwę oraz wartość w formie ciągu znaków.
\begin{lstlisting}[language=sh,alsoletter={!},keywords={!let,!prj,!files,!deps,!cflags,!pre,!post,!inc},keywordstyle=\bfseries]
!let project_name = "MyApp"
\end{lstlisting}

\paragraph{\texttt{!prj}} Definiuje nazwę projektu. Może zawierać bezpośrednią wartość lub odwołanie do zmiennej.
\begin{lstlisting}[language=sh,alsoletter={!},keywords={!let,!prj,!files,!deps,!cflags,!pre,!post,!inc}]
!prj "MyApp"
!prj $project_name
\end{lstlisting}

\paragraph{\texttt{!files}} Określa listę plików źródłowych do skompilowania. Pliki są umieszczone w nawiasach kwadratowych, oddzielone przecinkami i ujęte w cudzysłowy.
\begin{lstlisting}[language=sh,alsoletter={!},keywords={!let,!prj,!files,!deps,!cflags,!pre,!post,!inc}]
!files [ "main.cpp", "utils.cpp", "app.cpp" ]
\end{lstlisting}

\paragraph{\texttt{!deps}} Definiuje projekty zależne, które powinny zostać zbudowane przed głównym projektem. Musi być podana względna ścieżka do tych projektów.
\begin{lstlisting}[language=sh,alsoletter={!},keywords={!let,!prj,!files,!deps,!cflags,!pre,!post,!inc}]
!deps [ "dependency" ]
\end{lstlisting}

\paragraph{\texttt{!cflags}} Określa flagi kompilatora, takie jak opcje optymalizacji, ostrzeżenia czy inne parametry.
\begin{lstlisting}[language=sh,alsoletter={!},keywords={!let,!prj,!files,!deps,!cflags,!pre,!post,!inc}]
!cflags "-O2 -Wall -lsomelib"
\end{lstlisting}

\paragraph{\texttt{!pre}} Definiuje polecenia do wykonania przed rozpoczęciem procesu kompilacji, np. wyświetlenie komunikatu w terminalu.
\begin{lstlisting}[language=sh,alsoletter={!},keywords={!let,!prj,!files,!deps,!cflags,!pre,!post,!inc}]
!pre "echo 'Rozpoczynam budowanie projektu...'"
\end{lstlisting}

\paragraph{\texttt{!post}} Definiuje polecenia wykonywane po zakończeniu procesu kompilacji, np. informowanie o sukcesie lub kopiowanie plików.
\begin{lstlisting}[language=sh,alsoletter={!},keywords={!let,!prj,!files,!deps,!cflags,!pre,!post,!inc}]
!post "echo 'Budowanie zakończone!'"
\end{lstlisting}

\paragraph{\texttt{!inc}} Określa ścieżki do katalogów z plikami nagłówkowymi wymaganymi w procesie kompilacji.
\begin{lstlisting}[language=sh,alsoletter={!},keywords={!let,!prj,!files,!deps,!cflags,!pre,!post,!inc}]
!inc [ "include/", "external/includes/" ]
\end{lstlisting}

Każde z tych słów kluczowych jest integralną częścią procesu budowy projektu w BBS, umożliwiając elastyczną kontrolę nad kompilacją i konfiguracją projektu. Przygotowując plik konfiguracyjny zgodnie z opisanymi wytycznymi, można w pełni wykorzystać możliwości tego narzędzia.

\section{Działanie aplikacji}
Po uruchomieniu aplikacja odczytuje plik konfiguracyjny i realizuje zdefiniowane w nim kroki. Proces ten obejmuje:

\begin{itemize}
    \item Sprawdzenie, czy wszystkie pliki źródłowe i katalogi istnieją.
    \item Przygotowanie flag i parametrów kompilatora.
    \item Kompilację plików źródłowych i ich linkowanie do postaci programu wykonywalnego.
    \item Wykonanie poleceń określonych w sekcjach pre i post, jeśli zostały podane.
\end{itemize}

Jeśli podczas budowania projektu wystąpią błędy, takie jak brak plików źródłowych lub nieprawidłowa składnia w pliku konfiguracyjnym, aplikacja zgłasza je w formie komunikatów błędów w terminalu.